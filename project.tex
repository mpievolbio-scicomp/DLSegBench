%        File: project.tex
%     Created: Mon Aug 30 10:00 AM 2021 C
% Last Change: Mon Aug 30 10:00 AM 2021 C
%
\documentclass[a4paper]{article}
\usepackage[style=nature]{biblatex}
\addbibresource{references.bib}

\begin{document}

\section{Master thesis project: Deep Learning for Biological Image Segmentation\label{dlsegbench}}

\begin{description}
    \item[Hosting institution] Max Planck Institute for Evolutionary Biology, Dept. Microbial Population Biology (Prof.
        Paul Rainey)
    \item[Advisor] Dr. Carsten Fortmann-Grote
\end{description}


\subsection{Background}\label{project-description}

Image segmentation is a cornerstone of quantitative image analysis in
biomedical applications. Recently, deep learning \cite{Aggarwal2018} has become the de-facto
standard backbone technology \cite{Valen2016}. 

Deep learning based image segmentation is implemented in numerous open
source and commercial software packages. A non--exhaustive list is given below.
In the context of microbial
population biology, which is our research field, most available
implementations are trained on \textit{Escherichia coli} or \textit{Bacillus subtilis}
data. For our model organism, \textit{Pseudomonas fluorescens} strain SBW25,
neither training datasets, nor pre-trained models are available.
Therefore, we face the problem of having to filter and compare the available implementations and
identify the implementation or combination of implementations that work best for our datasets. 

\subsection{Image segmentation codes}
The following is a list of image segmentation codes that we wish to
compare:

\begin{enumerate}
    \def\labelenumi{\arabic{enumi}.}
    \item
        SuperSegger \cite{Stylianidou2016}
    \item
        Orbit \cite{Stritt2020}
    \item
        CellProfiler \cite{Jones2008}
    \item
        QuPath \cite{Bankhead2017}
\end{enumerate}

Some of these implementations come with pre-trained models for \textit{E. coli} or \textit{Bac. subtilis} data and provide means to
transfer or re-train these models on other data. Other implementations support manual annotation (labeling) of a subset of data and training the neural network from scratch.

\subsection{Image data}
Image data for \textit{P. fluorescens} under various growth conditions and using various microscopy techniques is readily
available in the department. More images can be provided if needed. Some data preparation,
such as shaping, normalization and centralization may be needed but is usually part of the implementation.

\subsection{Project outline}

The project can be divided into the following steps:
\begin{enumerate}
    \item Curate a model training dataset from recently
          obtained fluorescens microscopy timeseries taken from growing microbial cell cultures.
    \item Split the dataset into
        \begin{itemize}
            \item a \emph{training} dataset for iterative neural network training with the backpropagation
                algorithm 
            \item a \emph{validation} dataset to monitor the training
            \item a \emph{test} dataset to assess the model performance and accuracy.
        \end{itemize}
    \item Manual labeling the test dataset
    \item Identification of performance metric(s).
    \item Execution of training-validation-testing for the implementations listed above 
        and optional hyperparameter tuning.
    \item Analysis of training, validation, and test results
    \item Identification of the best model parameters and deep learning image segmentation implementation
    \item Documentation and publication of the study. 
\end{enumerate}

\subsection{Outlook}
Having curated the best segmentation implementation and model parameters for \textit{Pseudomonas fluorescens} SBW25 data, we can
then use the model to fine tune parameters for cell tracking and morphotyping algorithms, e.g. in SchnitzCell \cite{Young2012}. 

\printbibliography

\end{document}

%        File: project.tex
%     Created: Mon Aug 30 10:00 AM 2021 C
% Last Change: Mon Aug 30 10:00 AM 2021 C
%
\documentclass[a4paper]{article}
\usepackage[style=nature]{biblatex}
\addbibresource{jabref.db.bib}

\begin{document}

\section{Master thesis project: Deep Learning for Image Segmentation\label{dlsegbench}}

Benchmarking deep learning based cell segmentation

\subsection{Project description}\label{project-description}

Image segmentation is a cornerstone of quantitative image analysis in
biomedical applications. Recently, deep learning \cite{Aggarwal2018} has become the de-facto
standard backbone technology \cite{Valen2016}. 

Deep learning based image segmentation is implemented in numerous open
source and commercial software packages. A non-exhaustive list is given below.
In the context of microbial
population biology, which is our research field, most available
implementations are trained on Escherichia coli or Bacillus subtilis
data. For our model organism, Pseudomonas fluorescens strain SBW25,
neither training datasets, nor pre-trained models are available.
Therefore, we face the problem of having to filter and compare the available implementations and
identify the implementation or combination of implementations that work best for our datasets. 

The following is a list of image segmentation codes that we wish to
compare:

\begin{enumerate}
    \def\labelenumi{\arabic{enumi}.}
    \item
        SuperSegger \cite{Stylianidou2016}
    \item
        Orbit \cite{Stritt2020}
    \item
        CellProfiler \cite{Jones2008}
    \item
        QuPath \cite{Bankhead2017}
\end{enumerate}

Some of these implementations come with pre-trained models for E. coli or Bac. subtilis data and provide means to
transfer or re-train these models on other data. Other implementations support manual annotation (labeling) of a subset of data and training the neural network from scratch.

The project can be divided into the following steps:
\begin{enumerate}
    \item Curate a model training dataset using recently
          obtained fluorescens microscopy timeseries taken from growing microbial cell cultures.
    \item Split the dataset into
        \begin{itemize}
            \item a \emph{training} dataset for iterative neural network training with the backpropagation
                algorithm 
            \item a \emph{validation} dataset to monitor the training
            \item a \emph{test} dataset to assess the model performance and accuracy.
        \end{itemize}
    \item Manual labeling the test dataset
    \item Identification of performance metric(s).
    \item Execution of training-validation-testing for all candidate implementations and optional hyperparameter tuning.
    \item Analysis of training, validation, and test results
    \item Identification of the best model parameters and deep learning image segmentation implementation
    \item Documentation and publication of the study. 
\end{enumerate}

\subsection{Outlook}
Having curated the best segmentation implementation and model parameters for Pseudomonas fluorescens SBW25 data, we can
then use the model to fine tune parameters for cell tracking and morphotyping algorithms, e.g. in SchnitzCell \cite{Young2012}. 

\printbibliography

\end{document}
